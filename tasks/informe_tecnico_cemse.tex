\documentclass[12pt,a4paper]{article}
\usepackage[utf8]{inputenc}
\usepackage[spanish]{babel}
\usepackage{geometry}
\usepackage{graphicx}
\usepackage{fancyhdr}
\usepackage{titlesec}
\usepackage{enumitem}
\usepackage{url}
\usepackage{xcolor}
\usepackage{tcolorbox}

% Configuración de página
\geometry{margin=2.5cm}
\pagestyle{fancy}
\fancyhf{}
\fancyhead[L]{Informe Técnico - Emplea y Emprende}
\fancyhead[R]{\thepage}
\renewcommand{\headrulewidth}{0.4pt}

% Configuración de títulos
\titleformat{\section}{\Large\bfseries\color{blue!70!black}}{\thesection}{1em}{}
\titleformat{\subsection}{\large\bfseries\color{blue!50!black}}{\thesubsection}{1em}{}

\begin{document}

% Portada
\begin{titlepage}
    \centering
    \vspace*{2cm}

    {\Huge\bfseries Informe Técnico\\[0.5cm]}
    {\LARGE Plataforma Educativa Emplea y Emprende\\[2cm]}

    \begin{tcolorbox}[colback=blue!5!white,colframe=blue!75!black,width=0.8\textwidth]
        \centering
        \large
        \textbf{Sistema Integral de Gestión Educativa}\\
        Cursos, Instituciones y Empleo Juvenil
    \end{tcolorbox}

    \vfill

    {\large Versión: 0.1.0\\[0.5cm]}
    {\large Fecha: \today}

\end{titlepage}

\newpage
\tableofcontents
\newpage

\section{Resumen Ejecutivo}

Emplea y Emprende es una plataforma educativa integral desarrollada con tecnologías modernas que conecta instituciones educativas, empresas y jóvenes estudiantes. El sistema facilita la gestión de cursos, búsqueda de empleo, desarrollo de emprendimiento y administración institucional a través de una interfaz web unificada.

\section{Arquitectura del Sistema}

\subsection{Stack Tecnológico}

El sistema está construido sobre las siguientes tecnologías principales:

\begin{itemize}
    \item \textbf{Frontend:} Next.js 15 con React 19 y TypeScript
    \item \textbf{Base de Datos:} PostgreSQL con Prisma ORM
    \item \textbf{Caché:} Redis para sesiones y optimización
    \item \textbf{Almacenamiento:} MinIO para archivos y contenido multimedia
    \item \textbf{Autenticación:} NextAuth.js con control de acceso basado en roles
    \item \textbf{Contenedorización:} Docker y Docker Compose
    \item \textbf{Estilos:} Tailwind CSS con componentes Radix UI
\end{itemize}

\subsection{Estructura del Proyecto}

\begin{verbatim}
src/
├── app/                    # Rutas de Next.js App Router
│   ├── (auth)/            # Páginas de autenticación
│   ├── (dashboard)/       # Páginas del panel principal
│   └── api/               # Endpoints de API
├── components/            # Componentes React reutilizables
├── hooks/                 # Hooks personalizados de React
├── lib/                   # Utilidades y servicios
└── types/                 # Definiciones de tipos TypeScript
\end{verbatim}

\section{Módulos Principales}

\subsection{Sistema de Roles}

La plataforma maneja cuatro tipos de usuarios principales:

\begin{enumerate}
    \item \textbf{YOUTH (Jóvenes):} Estudiantes que buscan formación y empleo
    \item \textbf{COMPANIES (Empresas):} Organizaciones que publican ofertas laborales
    \item \textbf{INSTITUTION (Instituciones):} Centros educativos que ofrecen cursos
    \item \textbf{SUPERADMIN (Administradores):} Gestión global del sistema
\end{enumerate}

\subsection{Gestión de Cursos}

\begin{itemize}
    \item Creación y administración de contenido educativo
    \item Sistema de categorías y etiquetas
    \item Seguimiento de progreso estudiantil
    \item Generación de certificados
    \item Herramientas de evaluación (quizzes)
    \item Foros de discusión
\end{itemize}

\subsection{Portal de Empleo}

\begin{itemize}
    \item Publicación de ofertas laborales
    \item Sistema de aplicaciones y candidaturas
    \item Filtros avanzados de búsqueda
    \item Gestión de perfiles profesionales
    \item Seguimiento de aplicaciones
    \item Programación de entrevistas
\end{itemize}

\subsection{Emprendimiento}

\begin{itemize}
    \item Red social para emprendedores
    \item Calculadoras de negocio
    \item Sistema de conexiones y networking
    \item Analíticas de emprendimiento
    \item Recursos y noticias especializadas
\end{itemize}

\subsection{Administración Institucional}

\begin{itemize}
    \item Dashboard específico por institución
    \item Gestión de programas académicos
    \item Administración de estudiantes y docentes
    \item Reportes y analíticas educativas
    \item Gestión de inscripciones
\end{itemize}

\subsection{Sistema de Archivos}

\begin{itemize}
    \item Carga de archivos multimedia
    \item Gestión de documentos PDF
    \item Almacenamiento seguro con MinIO
    \item Sistema de thumbnails automático
    \item Carga chunked para archivos grandes
\end{itemize}

\section{Funcionalidades Clave}

\subsection{Autenticación y Seguridad}

\begin{itemize}
    \item Sistema de login/registro seguro
    \item Recuperación de contraseñas
    \item Control de acceso basado en roles
    \item Sesiones seguras con NextAuth.js
\end{itemize}

\subsection{Búsqueda Global}

\begin{itemize}
    \item Motor de búsqueda unificado
    \item Sugerencias automáticas
    \item Búsquedas populares
    \item Filtros contextuales
\end{itemize}

\subsection{Analíticas y Reportes}

\begin{itemize}
    \item Dashboard de métricas generales
    \item Analíticas de uso por módulo
    \item Gráficos de engagement
    \item Reportes de rendimiento
    \item Métricas demográficas
\end{itemize}

\subsection{Comunicación}

\begin{itemize}
    \item Sistema de mensajería interna
    \item Chat para entrevistas laborales
    \item Notificaciones en tiempo real
    \item Gestión de contenido y noticias
\end{itemize}

\section{Configuración y Despliegue}

\subsection{Desarrollo}

Para ejecutar el proyecto en desarrollo:

\begin{verbatim}
# Instalar dependencias
pnpm install

# Configurar variables de entorno
cp env.template .env

# Iniciar servicios de infraestructura
pnpm run docker:up

# Iniciar aplicación
pnpm run dev
\end{verbatim}

\subsection{Producción}

El sistema utiliza Docker Compose para despliegue en producción con:

\begin{itemize}
    \item Contenedores separados para cada servicio
    \item Configuración de variables de entorno específicas
    \item Balanceador de carga y SSL configurables
    \item Backups automatizados de base de datos
\end{itemize}

\section{Puntos de Acceso}

\subsection{Desarrollo}
\begin{itemize}
    \item \textbf{Aplicación:} http://localhost:3000
    \item \textbf{Consola MinIO:} http://localhost:9001
    \item \textbf{Prisma Studio:} http://localhost:5555
\end{itemize}

\subsection{Producción}
\begin{itemize}
    \item \textbf{Aplicación:} https://dominio-produccion.com
    \item \textbf{Consola MinIO:} https://dominio-produccion.com:9001
\end{itemize}

\section{Testing y Calidad}

El proyecto incluye:

\begin{itemize}
    \item Tests unitarios con Jest
    \item Testing de componentes React
    \item Configuración de ESLint y Prettier
    \item Validación de tipos con TypeScript
    \item Cobertura de código automatizada
\end{itemize}

\section{Conclusiones}

Emplea y Emprende representa una solución integral para el ecosistema educativo, combinando formación académica, búsqueda de empleo y desarrollo emprendedor en una sola plataforma. Su arquitectura moderna y escalable permite adaptarse a las necesidades cambiantes del sector educativo y laboral.

La plataforma facilita la conexión entre todos los actores del ecosistema educativo-laboral, proporcionando herramientas especializadas para cada tipo de usuario mientras mantiene una experiencia coherente y profesional.

\newpage
\section{Anexos}

\subsection{A. Capturas de Pantalla de Módulos Principales}

\textit{Nota: Las capturas de pantalla de los diferentes módulos de la aplicación se incluirán en esta sección para proporcionar una referencia visual de las funcionalidades principales:}

\begin{itemize}
    \item Dashboard principal por tipo de usuario
    \item Módulo de gestión de cursos
    \item Portal de empleo y aplicaciones
    \item Sistema de emprendimiento
    \item Panel de administración institucional
    \item Herramientas de analíticas
    \item Gestión de archivos y contenido
\end{itemize}

\subsection{B. Diagramas de Arquitectura}

\textit{Esta sección puede incluir diagramas detallados de:}

\begin{itemize}
    \item Arquitectura de componentes
    \item Flujo de datos
    \item Modelo de base de datos
    \item Infraestructura de despliegue
\end{itemize}

\end{document}